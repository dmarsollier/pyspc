\setlength\LTleft{0pt}
\setlength\LTright{0pt}
\begin{longtable}{@{\extracolsep{\fill}}|p{0.2\textwidth} | p{0.15\textwidth} |p{0.25\textwidth} | p{0.3\textwidth}|@{}}

\hline
Instructions  & 
Bibliothèque  & 
Rôle & 
Chapitre sur le site
\\ \hline

\textbackslash caractère spécial
 & 
\ldots{}
 & 
Insére un caractère spécial
 & 
\href{https://pyspc.readthedocs.io/fr/latest/05-bases/02-variables_input_print.html}{Affectation
des variables et impression des résultats}
\\ \hline

\textbackslash n
 & 
\ldots{}
 & 
Retour à la ligne (chaîne de caractères)
 & 
\href{https://pyspc.readthedocs.io/fr/latest/05-bases/02-variables_input_print.html}{Affectation
des variables et impression des résultats}
\\ \hline

\textbackslash t   
 &
 \ldots{}
 & 
Insère une tabulation
 & 
\href{https://pyspc.readthedocs.io/fr/latest/05-bases/02-variables_input_print.html}{Affectation
des variables et impression des résultats}

\\ \hline

\#
 & 
\ldots{}
 & 
Indique un commentaire
 & 
\href{https://pyspc.readthedocs.io/fr/latest/05-bases/01-structure-programme.html\#l\%E2\%80\%99importance-du-commentaire}{Structure
d'un programme}
\\ \hline

\%
 & 
\ldots{}
 & 
Opérateur modulo
 & 
\href{https://pyspc.readthedocs.io/fr/latest/05-bases/04-operations_basiques.html}{Opérations
basiques}
\\ \hline

**
 & 
\ldots{}
 & 
Opérateur puissance
 & 
\href{https://pyspc.readthedocs.io/fr/latest/05-bases/04-operations_basiques.html}{Opérations
basiques}
\\ \hline

==
 & 
\ldots{}
 & 
Test d'égalité
 & 
\href{https://pyspc.readthedocs.io/fr/latest/05-bases/06-boucles.html}{Les
boucles}
\\ \hline

abs
 & 
\ldots{}
 & 
Calcule la valeur absolue
 & 
\href{https://pyspc.readthedocs.io/fr/latest/05-bases/04-operations_basiques.html}{Opérations
basiques}
\\ \hline

and
 & 
\ldots{}
 & 
Opérateur logique ET
 & 
\href{https://pyspc.readthedocs.io/fr/latest/05-bases/04-operations_basiques.html}{Opérations
basiques}
\\ \hline

animation
 & 
matplotlib.pyplot
 & 
Simule le défilement d'une courbe
 & 
\href{https://pyspc.readthedocs.io/fr/latest/05-bases/12-animation.html}{Comment
animer un graphique?}
\\ \hline

append
 & 
\ldots{}
 & 
Ajoute une valeur à une liste
 & 
\href{https://pyspc.readthedocs.io/fr/latest/05-bases/07-listes.html}{Les
listes}
\\ \hline

arange
 & 
numpy
 & 
Crée un tableau
 & 
\href{https://pyspc.readthedocs.io/fr/latest/05-bases/07-listes.html}{Les
listes}
\\ \hline

array
 & 
numpy
 & 
Crée un tableau
 & 
\href{https://pyspc.readthedocs.io/fr/latest/05-bases/07-listes.html}{Les
listes}
\\ \hline

arrow
 & 
matplotlib.pyplot
 & 
Trace un vecteur
 & 
\href{https://pyspc.readthedocs.io/fr/latest/05-bases/10-graphiques_partie_2.html}{Comment
créer un graphique-2?}
\\ \hline

axis
 & 
matplotlib.pyplot
 & 
Gère les bornes des axes
 & 
\href{https://pyspc.readthedocs.io/fr/latest/05-bases/10-graphiques_partie_1.html}{Comment
créer un graphique-1?}
\\ \hline

break
 & 
\ldots{}
 & 
Interrompt une boucle
 & 
\href{https://pyspc.readthedocs.io/fr/latest/05-bases/06-boucles.html}{Les
boucles}
\\ \hline

choice
 & 
random
 & 
Renvoie au hasard un élément d'une liste
 & 
\href{https://pyspc.readthedocs.io/fr/latest/05-bases/07-listes.html}{Les
listes}
\\ \hline

continue
 & 
\ldots{}
 & 
Saute une étape dans une boucle
 & 
\href{https://pyspc.readthedocs.io/fr/latest/05-bases/06-boucles.html}{Les
boucles}
\\ \hline

def
 & 
\ldots{}
 & 
Définit une fonction
 & 
\href{https://pyspc.readthedocs.io/fr/latest/05-bases/05-bases/03-fonctions.html}{Les
fonctions}
\\ \hline

elif
 & 
\ldots{}
 & 
Test ``sinon si''
 & 
\href{https://pyspc.readthedocs.io/fr/latest/05-bases/05-tests.html}{Les
tests}
\\ \hline

else
 & 
\ldots{}
 & 
Test « sinon »
 & 
\href{https://pyspc.readthedocs.io/fr/latest/05-bases/05-tests.html}{Les
tests}
\\ \hline

end
 & 
\ldots{}
 & 
Saut à la ligne dans les affichages
 & 
\href{https://pyspc.readthedocs.io/fr/latest/05-bases/02-variables_input_print.html}{Affectation
des variables et impression des résultats}
\\ \hline

figure
 & 
matplotlib.pyplot
 & 
Crée une fenêtre graphique
 & 
\href{https://pyspc.readthedocs.io/fr/latest/05-bases/10-graphiques_partie_1.html}{Comment
créer un graphique-1?}
\\ \hline

float
 & 
\ldots{}
 & 
Convertit vers un réel
 & 
\href{https://pyspc.readthedocs.io/fr/latest/05-bases/02-variables_input_print.html}{Affectation
des variables et impression des résultats}
\\ \hline

for\ldots{}in
 & 
\ldots{}
 & 
Boucle ``de .. à''
 & 
\href{https://pyspc.readthedocs.io/fr/latest/05-bases/06-boucles.html}{Les
boucles}
\\ \hline

format
 & 
\ldots{}
 & 
Formate les valeurs affichées
 & 
\href{https://pyspc.readthedocs.io/fr/latest/05-bases/02-variables_input_print.html}{Affectation
des variables et impression des résultats}
\\ \hline

global
 & 
\ldots{}
 & 
Définit une variable comme étant globale
 & 
\href{https://pyspc.readthedocs.io/fr/latest/05-bases/05-bases/03-fonctions.html}{Les
fonctions}
\\ \hline

grid
 & 
matplotlib.pyplot
 & 
Affiche une grille sur le graphique
 & 
\href{https://pyspc.readthedocs.io/fr/latest/05-bases/10-graphiques_partie_1.html}{Comment
créer un graphique-1?}
\\ \hline

if
 & 
\ldots{}
 & 
Test « si »
 & 
\href{https://pyspc.readthedocs.io/fr/latest/05-bases/05-tests.html}{Les
tests}
\\ \hline

import
 & 
\ldots{}
 & 
Importe des bibliothèques
 & 
\href{https://pyspc.readthedocs.io/fr/latest/05-bases/01-structure-programme.html}{Structure
d'un programme}
\\ \hline

in
 & 
\ldots{}
 & 
Vérifie si une valeur est présente
 & 
\href{https://pyspc.readthedocs.io/fr/latest/05-bases/05-tests.html}{Les
tests}
\\ \hline

input
 & 
\ldots{}
 & 
Lit ce qui est tapé sur le clavier
 & 
\href{https://pyspc.readthedocs.io/fr/latest/05-bases/02-variables_input_print.html}{Affectation
des variables et impression des résultats}
\\ \hline

insert
 & 
\ldots{}
 & 
Insère un élément dans une liste
 & 
\href{https://pyspc.readthedocs.io/fr/latest/05-bases/07-listes.html}{Les
listes}
\\ \hline

int
 & 
\ldots{}
 & 
Convertit vers un entier
 & 
\href{https://pyspc.readthedocs.io/fr/latest/05-bases/02-variables_input_print.html}{Affectation
des variables et impression des résultats}
\\ \hline

join
 & 
\ldots{}
 & 
Concatène les éléments d'une liste séparés par un séparateur
 & 
\href{https://pyspc.readthedocs.io/fr/latest/05-bases/07-listes.html}{Les
listes}
\\ \hline

legend
 & 
matplotlib.pyplot
 & 
Affiche les étiquettes d'un graphique
 & 
\href{https://pyspc.readthedocs.io/fr/latest/05-bases/10-graphiques_partie_1.html}{Comment
créer un graphique-1?}
\\ \hline

len
 & 
\ldots{}
 & 
Renvoie la taille d'une liste
 & 
\href{https://pyspc.readthedocs.io/fr/latest/05-bases/07-listes.html}{Les
listes}
\\ \hline


list
 & 
\ldots{}
 & 
Convertit une chaîne de caractères en liste de caractères
 & 
\href{https://pyspc.readthedocs.io/fr/latest/05-bases/07-listes.html}{Les
listes}
\\ \hline

min
 & 
\ldots{}
 & 
Renvoie le plus petit élément d'une liste
 & 
\href{https://pyspc.readthedocs.io/fr/latest/05-bases/07-listes.html}{Les
listes}
\\ \hline

max
 & 
\ldots{}
 & 
Renvoie le plus grand élément d'une liste
 & 
\href{https://pyspc.readthedocs.io/fr/latest/05-bases/07-listes.html}{Les
listes}
\\ \hline

MultipleLocator
 & 
matplotlib.ticker
 & 
Gradue les axes
 & 
\href{https://pyspc.readthedocs.io/fr/latest/05-bases/10-graphiques_partie_1.html}{Comment
créer un graphique-1?}
\\ \hline

np.linspace
 & 
numpy
 & 
Crée un tableau
 & 
\href{https://pyspc.readthedocs.io/fr/latest/05-bases/08-tableaux_numpy.html}{Tableaux numpy}
\\ \hline


np.arange
 &
 numpy
 & 
 Crée un tableau numpy de valeurs séquentielles
 & 
\href{https://pyspc.readthedocs.io/fr/latest/05-bases/08-tableaux_numpy.html}{Tableaux numpy}
\\ \hline


np.array
 &
 numpy
 & 
Convertir une liste en tableau numpy
 & 
\href{https://pyspc.readthedocs.io/fr/latest/05-bases/08-tableaux_numpy.html}{Tableaux numpy}
\\ \hline



not in
 & 
\ldots{}
 & 
Vérifie qu'une valeur ne soit pas présente
 & 
\href{https://pyspc.readthedocs.io/fr/latest/05-bases/05-tests.html}{Les
tests}
\\ \hline

open
 & 
\ldots{}
 & 
Ouvre un fichier
 & 
\href{https://pyspc.readthedocs.io/fr/latest/05-bases/09-fichiers-csv.html}{Comment
importer des données numériques ?}
\\ \hline

or
 & 
\ldots{}
 & 
Opérateur logique OU
 & 
\href{https://pyspc.readthedocs.io/fr/latest/05-bases/04-operations_basiques.html}{Opérations
basiques}
\\ \hline

plot
 & 
matplotlib.pyplot
 & 
Crée un graphique
 & 
\href{https://pyspc.readthedocs.io/fr/latest/05-bases/10-graphiques_partie_1.html}{Comment
créer un graphique-1?}
\\ \hline

polyfit
 & 
numpy
 & 
Modélise une courbe
 & 
\href{https://pyspc.readthedocs.io/fr/latest/05-bases/11-modelisation.html}{Comment
modéliser une courbe?}
\\ \hline

pop
 & 
\ldots{}
 & 
Supprime un élément d'une liste
 & 
\href{https://pyspc.readthedocs.io/fr/latest/05-bases/07-listes.html}{Les
listes}
\\ \hline

print
 & 
\ldots{}
 & 
Affiche à l'écran
 & 
\href{https://pyspc.readthedocs.io/fr/latest/05-bases/02-variables_input_print.html}{Affectation
des variables et impression des résultats}
\\ \hline

range
 & 
\ldots{}
 & 
Définit un intervalle
 & 
\href{https://pyspc.readthedocs.io/fr/latest/05-bases/06-boucles.html}{Les
boucles}
\\ \hline

reader
 & 
\ldots{}
 & 
Lit un fichier
 & 
\href{https://pyspc.readthedocs.io/fr/latest/05-bases/09-fichiers-csv.html}{Comment
importer des données numériques ?}
\\ \hline

remove
 & 
\ldots{}
 & 
Supprime un élément d'une liste
 & 
\href{https://pyspc.readthedocs.io/fr/latest/05-bases/07-listes.html}{Les
listes}
\\ \hline

return
 & 
\ldots{}
 & 
Renvoi d'une valeur par la fonction
 & 
\href{https://pyspc.readthedocs.io/fr/latest/05-bases/05-bases/03-fonctions.html}{Les
fonctions}
\\ \hline

round
 & 
\ldots{}
 & 
Arrondit une valeur
 & 
\href{https://pyspc.readthedocs.io/fr/latest/05-bases/02-variables_input_print.html}{Affectation
des variables et impression des résultats}
\\ \hline

sample
 & 
random
 & 
Renvoie au hasard un ou plusieurs éléments d'une liste
 & 
\href{https://pyspc.readthedocs.io/fr/latest/05-bases/07-listes.html}{Les
listes}
\\ \hline

sep
 & 
\ldots{}
 & 
Insère un séparateur entre plusieurs affichages
 & 
\href{https://pyspc.readthedocs.io/fr/latest/05-bases/02-variables_input_print.html}{Affectation
des variables et impression des résultats}
\\ \hline

show
 & 
matplotlib.pyplot
 & 
Affiche le graphique
 & 
\href{https://pyspc.readthedocs.io/fr/latest/05-bases/10-graphiques_partie_1.html}{Comment
créer un graphique-1?}
\\ \hline

sort
 & 
\ldots{}
 & 
Trie une liste
 & 
\href{https://pyspc.readthedocs.io/fr/latest/05-bases/07-listes.html}{Les
listes}
\\ \hline

sorted
 & 
\ldots{}
 & 
Trie une liste
 & 
\href{https://pyspc.readthedocs.io/fr/latest/05-bases/07-listes.html}{Les
listes}
\\ \hline

sqrt
 & 
\ldots{}
 & 
Calcule la racine carrée
 & 
\href{https://pyspc.readthedocs.io/fr/latest/05-bases/04-operations_basiques.html}{Opérations
basiques}
\\ \hline

subplot
 & 
matplotlib.pyplot
 & 
Divise la fenêtre graphique en plusieurs emplacements
 & 
\href{https://pyspc.readthedocs.io/fr/latest/05-bases/10-graphiques_partie_2.html}{Comment
créer un graphique-2?}
\\ \hline

subplots\_adjust
 & 
matplotlib.pyplot.gcf
 & 
Définit la position d'un graphique dans un emplacement
 & 
\href{https://pyspc.readthedocs.io/fr/latest/05-bases/10-graphiques_partie_2.html}{Comment
créer un graphique-2?}
\\ \hline

text
 & 
matplotlib.pyplot
 & 
Affiche un texte sur le graphique
 & 
\href{https://pyspc.readthedocs.io/fr/latest/05-bases/10-graphiques_partie_1.html}{Comment
créer un graphique-1?}
\\ \hline

title
 & 
\ldots{}
 & 
Nomme un graphique
 & 
\href{https://pyspc.readthedocs.io/fr/latest/05-bases/10-graphiques_partie_1.html}{Comment
créer un graphique-1?}
\\ \hline

type
 & 
\ldots{}
 & 
Indique le type d'une variable
 & 
\href{https://pyspc.readthedocs.io/fr/latest/05-bases/02-variables_input_print.html}{Affectation
des variables et impression des résultats}
\\ \hline

while
 & 
\ldots{}
 & 
Boucle « tant que »
 & 
\href{https://pyspc.readthedocs.io/fr/latest/05-bases/06-boucles.html}{Les
boucles}
\\ \hline

xlabel (ou ylabel)
 & 
matplotlib.pyplot
 & 
Légende les axes
 & 
\href{https://pyspc.readthedocs.io/fr/latest/05-bases/10-graphiques_partie_1.html}{Comment
créer un graphique-1?}
\\ \hline

xlim ( ou (ylim)
 & 
matplotlib.pyplot
 & 
Définit les bornes sur les axes
 & 
\href{https://pyspc.readthedocs.io/fr/latest/05-bases/10-graphiques_partie_1.html}{Comment
créer un graphique-1?}
\\ \hline
\end{longtable}
